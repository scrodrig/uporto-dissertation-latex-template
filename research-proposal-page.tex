%% FEUP THESIS STYLE for LaTeX2e
%% how to use feupteses (English version)
%%
%% FEUP, JCL & JCF, 31 July 2012
%%
%% Read the documentation inline and
%% at https://web.fe.up.pt/~jlopes/doku.php/teach/feupteses
%%
%% PLEASE send improvements to jlopes at fe.up.pt and to jcf at fe.up.pt
%%

%%========================================
%% Commands: pdflatex tese
%%           bibtex tese
%%           makeindex tese (only if creating an index)
%%           pdflatex tese
%% Alternative:
%%          latexmk -pdf tese.tex
%%========================================

%% 2021-07-20: One-sided output by default
\documentclass[11pt,a4paper]{report}

%% For two-sided printing (for dead-tree output) comment previous line
%% and uncomment the next line
%% \documentclass[11pt,a4paper,twoside,openright]{report}

%% For iso-8859-1 (latin1), comment next line and uncomment the second line
\usepackage[utf8]{inputenc}
%\usepackage[latin1]{inputenc}

%% English version

%% MEIC options
%\usepackage[meic]{feupteses}
%\usepackage[meic,juri]{feupteses}
%\usepackage[meic,final]{feupteses}
%\usepackage[meic,final,onpaper]{feupteses}

%% MEEC options
%\usepackage[meec]{feupteses}
%\usepackage[meec,juri]{feupteses}
%\usepackage[meec,final]{feupteses}

%% For other degrees
\usepackage{styles/feupteses} % you must define the degree bellow

%% Additional options for feupteses.sty: 
%% - portugues: titles, etc in portuguese
%% - onpaper: links are not shown (for paper versions)
%% - backrefs: include back references from bibliography to citation place

%% include my packages not included in feupteses.sty
%% 

%% Uncomment the next lines if side by side graphics used
\usepackage[lofdepth,lotdepth]{subfig}
\usepackage{graphicx}
\usepackage{float}
\usepackage{subfiles}

% Listings
\definecolor{cloudwhite}{cmyk}{0,0,0,0.025}  % color

%% Include source-code listings package
\usepackage{listings}
\lstset{ %
 language=C,                        % choose the language of the code
 basicstyle=\footnotesize\ttfamily,
 keywordstyle=\bfseries,
 numbers=left,                      % where to put the line-numbers
 numberstyle=\scriptsize\texttt,    % the size of the fonts that are used for the line-numbers
 stepnumber=1,                      % the step between two line-numbers. If it's 1 each line will be numbered
 numbersep=8pt,                     % how far the line-numbers are from the code
 frame=tb,
 float=htb,
 aboveskip=8mm,
 belowskip=4mm,
 backgroundcolor=\color{cloudwhite},
 showspaces=false,                  % show spaces adding particular underscores
 showstringspaces=false,            % underline spaces within strings
 showtabs=false,                    % show tabs within strings adding particular underscores
 tabsize=2,                         % sets default tabsize to 2 spaces
 captionpos=b,                      % sets the caption-position to bottom
 breaklines=true,                   % sets automatic line breaking
 breakatwhitespace=false,           % sets if automatic breaks should only happen at whitespace
 escapeinside={\%*}{*)},            % if you want to add a comment within your code
 morekeywords={*,var,template,new}  % if you want to add more keywords to the set
}

%% Uncomment to create an index (at the end of the document)
%\makeindex

%% Path to the figures directory
%% TIP: use folder ``figures'' to keep all your figures
\graphicspath{{figures/}}

%%========================================
%% Start of document
%%========================================
\begin{document}

%%----------------------------------------
%% TIP: if you want to define more macros, use an external file to keep them
\include{macros/mymacros}
%%----------------------------------------

%%----------------------------------------
%% Information about the work
%%----------------------------------------
\title{Title of the Dissertation}
\author{Author name}

%% Comment next line if not necessary for degree
\degree{Programa Doutoral em Engenharia Informática}

%% Uncomment next line for date of submission
%\thesisdate{July 31, 2008}

%% Comment next line copyright text if not used
\copyrightnotice{Author name, 2008}

\supervisor{Supervisor}{Name of the Supervisor}

%% Uncomment next line if necessary
%\supervisor{Second Supervisor}{Name of the Supervisor}

%% Uncomment committee stuff in the final version if used
%\committeetext{Approved by \ldots:}
%\committeemember{President}{Name of the President}
%\committeemember{Referee}{Name of the Referee}
%\committeemember{Referee}{Name of the Referee}

%% Uncomment signature line in the final on paper version if used
%\signature

%% Specify cover logo (in folder ``figures'')
\logo{uporto-feup.pdf}

%% Uncomment next line for additional text below the author's name (front page)
%\additionalfronttext{Preparação da Dissertação}

%%----------------------------------------
%% Preliminary materials
%%----------------------------------------

% remove unnecessary \include{} commands
\begin{Prolog}
  \include{abstract/abstract} % the abstract
  \include{ackowledges/acknows}  % the acknowledgments
  \include{quote/quote}    % initial quotation if desired
  \cleardoublepage
  \pdfbookmark[0]{Table of Contents}{contents}
  \tableofcontents
  \cleardoublepage
  \pdfbookmark[0]{List of Figures}{figures}
  \listoffigures
  \cleardoublepage
  \pdfbookmark[0]{List of Tables}{tables}
  \listoftables
  \include{abbrevations/abbrevs}  % the list of abbreviations used
\end{Prolog}

%%----------------------------------------
%% Body
%%----------------------------------------
\StartBody

%% TIP: use a separate file for each chapter
\include{chapters/chapter1}
\include{chapters/chapter2}
\include{chapters/chapter3}
\include{chapters/chapter4}
\include{chapters/chapter5}

%%----------------------------------------
%% Final materials
%%----------------------------------------

%% Bibliography
%% Comment the next command if BibTeX file not used
%% bibliography is in ``myrefs.bib''
\PrintBib{references/myrefs}

%% 2021-07-20: change
%% comment next 2 commands if numbered appendices are not used
\appendix
\include{appendix/appendix1}

%% Index
%% Uncomment next command if index is required
%% don't forget to run ``makeindex thesis'' command
%\PrintIndex

\end{document}










% \documentclass[a4paper,12pt]{report}
% \setlength{\parskip}{0.5pt}%
% \setlength{\parindent}{20pt}%

% %preamble: style and/or packages
% \author{Michele Magni}
% \title{PhD research proposal}
% \date{January 2023}

% %\usepackage{package}

% \usepackage{hyperref}
% \usepackage{titlesec}
% \setcounter{secnumdepth}{3}
% \usepackage{enumitem}
% \usepackage{varwidth}
% \usepackage{tasks}

% \usepackage{graphicx}
% \usepackage{siunitx}

% %colors, boxes
% \usepackage[dvipsnames]{xcolor}
% \usepackage[most]{tcolorbox}
% \tcbuselibrary{fitting}

% \definecolor{columbiablue}{rgb}{0.61, 0.87, 1.0}
% \definecolor{mossgreen}{rgb}{0.68, 0.87, 0.68}

% \usepackage[super,sort&compress,comma]{natbib}
% \bibliographystyle{naturemag}

% %indent first line
% \usepackage{indentfirst}
% \setlength{\parindent}{30pt}

% %captions
% \usepackage[font=footnotesize,labelfont={bf,it}, textfont=it]{caption}
% \usepackage[labelsep=period]{caption}

% %landscape pages
% \usepackage{pdflscape}
% \usepackage{fancyhdr} 
% \usepackage{feupphdteses} 

% %page number at bottom in landscape 
% \fancypagestyle{mylandscape}{
% \fancyhf{} %Clears the header/footer
% \fancyfoot{% Footer
% \makebox[\textwidth][r]{% Right
%   \rlap{\hspace{.75cm}% Push out of margin by \footskip
%     \smash{% Remove vertical height
%       \raisebox{4.87in}{% Raise vertically
%         \rotatebox{90}{\thepage}}}}}}% Rotate counter-clockwise
% \renewcommand{\headrulewidth}{0pt}% No header rule
% \renewcommand{\footrulewidth}{0pt}% No footer rule
% }

% \newcommand{\Feup}{Faculdade de Engenharia da Universidade do Porto}

% %temporarily disable superscript
% \DeclareRobustCommand*{\citen}[1]{%
%   \begingroup
%     \romannumeral-`\x % remove space at the beginning of \setcitestyle
%     \setcitestyle{numbers}%
%     \cite{#1}%
%   \endgroup   
% }


% \begin{document}

% % cover page
% \begin{center}
%   \thispagestyle{empty}
%   \begin{LARGE}
%     Title  \\[1 cm] \vfill
%   \end{LARGE}


%   \begin{Large}
%     PhD research proposal \\ [1 cm]\vfill
%     Author \\
%     \href{mailto:<email>}{$<$email$>$} \\[1 cm]\vfill


%     Month Year\\[1 cm]\vfill

%     1st supervisor: $<$1st supervisor$>$ \\
%     2nd supervisor: $<$2nd supervisor$>$ \\[1 cm]\vfill

%     Department of $<$Department$>$ \\
%     \Feup \\
%     %choose between bnw or rgb logo
%     \centerline{
%       \includegraphics[trim={0cm 1cm 2.1cm 0.5cm},clip]{./figures/UU_logo_2021_NL_RGB.jpg}
%     }

%   \end{Large}
% \end{center}

% % document begins
% \newpage
% %%table of contents 
% {\setlength\parskip{\fill}
%   \tableofcontents
% }

% %you can start a new page anytime with \newpage
% \newpage

% %Abstract
% \section*{Abstract}
% \addcontentsline{toc}{section}{\protect\numberline{}Abstract}

% %%Introduction
% \newpage
% \section{Introduction}

% \subsection{Background}

% \subsubsection{An additional subsection}

% The most cited paper \citep{lowry1951protein}.

% You can reference a figure like this (fig.~\ref{fig:mandelbrot}). Upload your own figures in the /figures/ folder.

% %figure template
% \begin{figure}[ht]
%   \centerline{
%     \includegraphics[scale=1.2]{./figures/mandelbrot.jpg}
%   }
%   \caption{A caption \citep{mandelbrot1982fractal} or non-superscripted reference [\citen{mandelbrot1982fractal}]}
%   \label{fig:mandelbrot}
% \end{figure}

% \newpage
% \subsection{State of the art}

% \begin{enumerate}
%   \item
%         First item;
%   \item
%         Second item;
%   \item
%         Third item.
% \end{enumerate}

% \subsection{Knowledge gaps}

% \begin{itemize}
%   \item
%         First item;
%   \item
%         Second item;
%   \item
%         Third item.
% \end{itemize}


% %Add extra space 
% \vspace*{5mm}

% %Objective and research questions
% \newpage
% \section{Objective and research questions}

% \begin{tcolorbox}[minipage,colback=Goldenrod,arc=0pt,outer arc=0pt]
%   \centering
%   \textbf{An overarching objective.}
% \end{tcolorbox}

% \textbf{RQ 1.}	\emph{First research question?}\\

% \textbf{RQ 2.}	\emph{Second research question?}\\

% \textbf{RQ 3.}	\emph{Et cetera...}\\

% %Methods
% \newpage
% \section{Methods}
% \subsection{Methods tackling the first research question}

% \vspace*{10mm}

% \begin{tcolorbox}[minipage,colback=columbiablue,arc=10pt,outer arc=10pt]
%   \centering
%   \textbf{RQ 1.}	\emph{First research question}
% \end{tcolorbox}

% \vspace*{10mm}


% \noindent \textbf{a. Methodology}\\ [0.1 cm]

% \textbf{i. First step}

% \textbf{ii. Second step}
% \vspace*{5mm}

% %example of equation
% An equation (eq. \ref{eq:euler}):
% \begin{equation} \label{eq:euler}
%   e^{ \pm i\theta } = \cos \theta \pm i\sin \theta
% \end{equation}

% \textbf{iii. Third step}
% \vspace*{5mm}


% \noindent \textbf{b. Novelty}\\ [0.1 cm]

% \begin{tcolorbox}[minipage,colback=mossgreen,arc=10pt,outer arc=10pt]
%   \centering
%   \textbf{Paper 1. \emph{First paper}}
% \end{tcolorbox}

% \vspace*{5mm}

% \noindent \textbf{c. Risks and feasibility}\\ [0.1 cm]

% \noindent \textbf{d. Outcomes} \vspace*{5mm}

% \textbf{Outcome 1.1.} Outcome 1.1\\

% \textbf{Outcome 1.2.} Outcome 1.2.



% % you can find the template for this gantt chart in /figures/gantt_template.pptx, convert it to pdf and then to png to upload it on the document

% %example of a temporarily horizontal page
% \newpage
% \begin{landscape}
%   \thispagestyle{mylandscape}
%   \vspace*{-3cm}
%   \section{Timetable}
%   \begin{figure}[!htb]\vspace*{-0.4cm}
%     \centerline{
%       \includegraphics[scale=0.67]{./figures/gantt_template.png}
%     }
%     \caption{Gantt chart of the PhD project.}
%     \label{fig:gantt}
%   \end{figure}
% \end{landscape}



% \newpage
% \section{Collaborations}
% \vspace*{1 cm}
% \noindent \textbf{University - Department}\\

% \noindent University address\\

% \textbf{Theme}

% Person 1 - \href{mailto:<email>}{$<$email$>$} \\ \vspace*{-5.5mm}

% Person 2 – \href{mailto:<email>}{$<$email$>$} \\

% \textbf{Theme 2}

% Person 1 - \href{mailto:<email>}{$<$email$>$} \\ \vspace*{-5.5mm}

% Person 2 – \href{mailto:<email>}{$<$email$>$} \\\\


% \section{Data management plan}

% \newpage
% % references
% \addcontentsline{toc}{section}{References}
% \bibliography{references.bib}

% \end{document}

